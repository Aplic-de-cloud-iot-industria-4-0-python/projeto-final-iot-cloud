\documentclass{estacio}

\usepackage{listings} % Pacote para inclusão de códigos

% Configuração do ambiente para códigos Python
\lstnewenvironment{pythoncode}[1][]{
    \lstset{
        language=Python,
        basicstyle=\scriptsize\ttfamily, % Altere para \scriptsize ou \small
        keywordstyle=\color{blue!70!black}\bfseries,
        stringstyle=\color{orange!70!black},
        commentstyle=\color{green!70!black}\itshape,
        morecomment=[l][\color{magenta}]{\#},
        numbers=left,
        numberstyle=\tiny\color{gray}, % Tamanho das linhas numeradas
        stepnumber=1,
        numbersep=5pt,
        tabsize=4,
        showspaces=false,
        showstringspaces=false,
        breaklines=true,
        frame=tb,
        framexleftmargin=5mm,
        aboveskip=3mm,
        belowskip=3mm,
        captionpos=b,
        #1
    }
}{}

\begin{document}

\maketitle

\section{Introdução}
Arduino é uma plataforma de prototipagem eletrônica de código aberto que consiste em hardware e software flexíveis, amplamente utilizada por entusiastas, estudantes e profissionais em áreas como engenharia, design e programação. Existem diversos tipos de placas Arduino, cada uma com características técnicas específicas que se adequam a diferentes necessidades e projetos.

\section{Principais Tipos de Arduino}

\subsection{Arduino Uno}
\subsubsection{Características técnicas}
\begin{itemize}
    \item Microcontrolador: ATmega328P.
    \item Memória flash: 32KB (dos quais 0.5KB são usados pelo bootloader).
    \item SRAM: 2KB.
    \item Clock: 16MHz.
\end{itemize}
\subsubsection{Aplicações}
\begin{itemize}
    \item Projetos iniciantes.
    \item Projetos de aprendizado.
    \item Prototipagem rápida.
\end{itemize}
\subsubsection{Benefícios}
\begin{itemize}
    \item Fácil de usar e programar.
    \item Preço acessível.
    \item Comunidade ativa e vasta documentação disponível.
\end{itemize}
\subsubsection{Concorrentes}
\begin{itemize}
    \item Arduino Nano.
    \item Arduino Leonardo.
\end{itemize}

\subsection{Arduino Mega}
\subsubsection{Características técnicas}
\begin{itemize}
    \item Microcontrolador: ATmega2560.
    \item Memória flash: 256KB (dos quais 8KB são usados pelo bootloader).
    \item SRAM: 8KB.
    \item Clock: 16MHz.
\end{itemize}
\subsubsection{Aplicações}
\begin{itemize}
    \item Projetos complexos que exigem mais pinos de I/O e mais memória.
    \item Automação residencial.
    \item Robótica avançada.
\end{itemize}
\subsubsection{Benefícios}
\begin{itemize}
    \item Grande quantidade de pinos de I/O.
    \item Memória flash e SRAM expandidas.
    \item Suporte para projetos exigentes.
\end{itemize}
\subsubsection{Concorrentes}
\begin{itemize}
    \item Arduino Due.
    \item Placas compatíveis com ARM Cortex.
\end{itemize}

\subsection{Arduino Nano}
\subsubsection{Características técnicas}
\begin{itemize}
    \item Microcontrolador: ATmega328P.
    \item Memória flash: 32KB (dos quais 2KB são usados pelo bootloader).
    \item SRAM: 2KB.
    \item Clock: 16MHz.
\end{itemize}
\subsubsection{Aplicações}
\begin{itemize}
    \item Projetos compactos.
    \item Projetos wearable.
    \item Aplicações com restrições de espaço.
\end{itemize}
\subsubsection{Benefícios}
\begin{itemize}
    \item Tamanho compacto.
    \item Pode ser soldado diretamente em placas de circuito impresso.
    \item Adequado para projetos portáteis.
\end{itemize}
\subsubsection{Concorrentes}
\begin{itemize}
    \item Arduino Pro Mini.
    \item Adafruit Trinket.
\end{itemize}

\section{Considerações Finais}
Cada tipo de Arduino oferece vantagens específicas, desde a facilidade de uso e baixo custo até a capacidade de lidar com projetos complexos e exigentes. A escolha entre os tipos de Arduino depende das necessidades do projeto, do orçamento e das preferências pessoais do usuário. A ampla disponibilidade de documentação, tutoriais e comunidade de suporte torna a plataforma Arduino uma escolha popular para uma variedade de aplicações de prototipagem e desenvolvimento.

\end{document}