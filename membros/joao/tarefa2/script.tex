%%%%%%%%%%%%%%%%%%%%%%%%%%%%%%%%%%%%%%%%%%%%%
%                Rubens Braz                %
%            www.rubensbraz.com             %
%%%%%%%%%%%%%%%%%%%%%%%%%%%%%%%%%%%%%%%%%%%%%

% Definições Iniciais --------------------------------------------------
\documentclass[12pt]{report}

%\usepackage[a4paper]{geometry}
\usepackage[left=2.8cm, right=2.8cm, top=4cm]{geometry} % margens da folha

\usepackage[utf8]{inputenc} % caracteres com acento

\usepackage[english,portuguese]{babel}
\usepackage[T1]{fontenc} %tradução

\usepackage{url} % exibição de URL's

\usepackage{subfigure}
\usepackage{graphicx, multicol, latexsym, amsmath, amssymb} % permite inserção de imagens e fórmulas matemáticas

\graphicspath{{../pdf/}{../jpeg/}}
\DeclareGraphicsExtensions{.pdf,.jpeg,.png} % formatos aceitos

\usepackage{hyphenat} % hífens; veja a última linha antes do início do documento, corrigir erros de hifenização lá!

\usepackage{etoolbox}
\apptocmd{\sloppy}{\hbadness 10000\relax}{}{} % remove erros de underfull hbox em links

\usepackage{color} % permite a mudança de cor do texto

\usepackage{xcolor,soul,framed}
\colorlet{shadecolor}{yellow} % define cor do destacado como amarelo

\usepackage{silence}
\WarningFilter{caption}{Unsupported document class} % remove o erro no pacote 'caption'
\usepackage{caption} % permite a inserção de legendas em imagens

\usepackage{eqparbox} % facilita a criação de grupos

\usepackage{listings} % permite a escrita de códigos de programação

\usepackage{indentfirst} % identação depois de uma seção

\usepackage{tabularx} % tabelas com tamanho variável de coluna

\usepackage{array}

\makeatletter
\patchcmd{\chapter}{\if@openright\cleardoublepage\else\clearpage\fi}{}{}{}
\makeatother % Manter capítulos na mesma página

\hyphenation{matemá-tica} % corrigir erros de hifenização aqui!

\usepackage{placeins}

% Início do documento --------------------------------------------------
\begin{document}

% Cabeçalho
\begin{center}
    \vspace*{-3cm}
    \textsc{Universidade Estacio} \\
    \textsc{Sistemas de Informação}
    
    \vspace{1cm}
    \rule{411pt}{1.3pt}
    \vspace{0.2cm}
    
    \Large \textbf{\textsc{Produto Desenvolvimento}}
    
    \rule{411pt}{1.3pt}
    \vspace{1cm}
    
    Aluno: João Daniel da Silva\\
    \vspace{0.5cm}
  
    
    \vspace{1cm}
\end{center}


% Resumo
\documentclass{article}
\usepackage[utf8]{inputenc}

\title{Relatório Técnico de Produtos IoT}
\author{Seu Nome}
\date{\today}

\begin{document}

\maketitle

\documentclass{article}
\usepackage[utf8]{inputenc}
\usepackage{xcolor}
\usepackage{graphicx}

% Definindo cores personalizadas
\definecolor{arduino_blue}{RGB}{0, 105, 178}
\definecolor{arduino_green}{RGB}{40, 145, 40}
\definecolor{arduino_red}{RGB}{191, 37, 45}
\definecolor{arduino_yellow}{RGB}{254, 221, 0}
\definecolor{arduino_purple}{RGB}{123, 50, 148}

\title{Tipos de Arduino e suas Diferenças}
\author{Seu Nome}
\date{\today}

\begin{document}

\maketitle

\section{Tipos de Arduino}

\subsection{Arduino Uno}
\begin{center}
\textcolor{arduino_blue}{\textbf{Características Técnicas}}
\begin{itemize}
    \item Processador: ATmega328P
    \item Pinos de E/S Digitais: 14
    \item Pinos de E/S Analógicas: 6
    \item Memória Flash: 32KB
\end{itemize}

\textcolor{arduino_blue}{\textbf{Aplicações}}
\begin{itemize}
    \item Projetos iniciantes
    \item Automação residencial simples
    \item Controle de dispositivos básicos
\end{itemize}

\textcolor{arduino_blue}{\textbf{Benefícios}}
\begin{itemize}
    \item Fácil de usar e aprender
    \item Ampla comunidade e documentação
    \item Preço acessível
\end{itemize}
\end{center}

\subsection{Arduino Mega}
\begin{center}
\textcolor{arduino_green}{\textbf{Características Técnicas}}
\begin{itemize}
    \item Processador: ATmega2560
    \item Pinos de E/S Digitais: 54
    \item Pinos de E/S Analógicas: 16
    \item Memória Flash: 256KB
\end{itemize}

\textcolor{arduino_green}{\textbf{Aplicações}}
\begin{itemize}
    \item Projetos complexos
    \item Robótica
    \item Automação industrial
\end{itemize}

\textcolor{arduino_green}{\textbf{Benefícios}}
\begin{itemize}
    \item Grande número de pinos de I/O
    \item Compatibilidade com shields Arduino Uno
    \item Ideal para projetos avançados
\end{itemize}
\end{center}

\subsection{Arduino Nano}
\begin{center}
\textcolor{arduino_red}{\textbf{Características Técnicas}}
\begin{itemize}
    \item Processador: ATmega328P
    \item Pinos de E/S Digitais: 14
    \item Pinos de E/S Analógicas: 8
    \item Memória Flash: 32KB
\end{itemize}

\textcolor{arduino_red}{\textbf{Aplicações}}
\begin{itemize}
    \item Projetos compactos
    \item Wearables
    \item Aplicações embarcadas
\end{itemize}

\textcolor{arduino_red}{\textbf{Benefícios}}
\begin{itemize}
    \item Tamanho compacto
    \item Baixo consumo de energia
    \item Facilidade de integração em projetos pequenos
\end{itemize}
\end{center}

% Adicione mais tipos de Arduino aqui






% Bibliografia
\documentclass{article}
\usepackage[utf8]{inputenc}

\title{Bibliografia de IoT}
\author{Seu Nome}
\date{\today}

\begin{document}

\maketitle

\section{Bibliografia de IoT}

Nos últimos anos, o desenvolvimento de Internet das Coisas (IoT) tem crescido exponencialmente, impulsionado pela demanda por soluções inteligentes e conectadas em diversos setores. A IoT permite a interconexão de dispositivos físicos, coleta de dados em tempo real e automação de processos, trazendo benefícios significativos para residências, empresas e indústrias.

O desenvolvimento de IoT abrange várias etapas, desde a concepção e prototipagem de dispositivos até a implementação de sistemas de gerenciamento de dados e análise de dados. Algumas das principais áreas de foco incluem:

\begin{itemize}
    \item Desenvolvimento de hardware: Projeto e fabricação de dispositivos inteligentes, sensores e atuadores com capacidade de comunicação.
    \item Desenvolvimento de software: Programação de firmware para dispositivos IoT, desenvolvimento de aplicativos móveis e web para interação com os dispositivos.
    \item Conectividade e comunicação: Implementação de protocolos de comunicação como MQTT, CoAP e HTTP para permitir a troca de dados entre dispositivos e servidores.
    \item Segurança: Incorporação de medidas de segurança, como criptografia e autenticação, para proteger os dados transmitidos e garantir a integridade dos dispositivos.
    \item Integração de sistemas: Desenvolvimento de plataformas e middleware para integrar dispositivos IoT com sistemas legados e outros serviços na nuvem.
\end{itemize}

O desenvolvimento de IoT é um campo multidisciplinar que requer conhecimentos em eletrônica, programação, redes de computadores, segurança da informação e design de sistemas. Com o avanço da tecnologia e a crescente adoção de dispositivos conectados, espera-se que o desenvolvimento de IoT continue a evoluir e gerar impactos significativos em diferentes aspectos da vida moderna.



\end{document}
