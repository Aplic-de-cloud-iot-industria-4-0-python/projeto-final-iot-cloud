%%%%%%%%%%%%%%%%%%%%%%%%%%%%%%%%%%%%%%%%%%%%%
%                Rubens Braz                %
%            www.rubensbraz.com             %
%%%%%%%%%%%%%%%%%%%%%%%%%%%%%%%%%%%%%%%%%%%%%

% Definições Iniciais --------------------------------------------------
\documentclass[12pt]{report}

%\usepackage[a4paper]{geometry}
\usepackage[left=2.8cm, right=2.8cm, top=4cm]{geometry} % margens da folha

\usepackage[utf8]{inputenc} % caracteres com acento

\usepackage[english,portuguese]{babel}
\usepackage[T1]{fontenc} %tradução

\usepackage{url} % exibição de URL's

\usepackage{subfigure}
\usepackage{graphicx, multicol, latexsym, amsmath, amssymb} % permite inserção de imagens e fórmulas matemáticas

\graphicspath{{../pdf/}{../jpeg/}}
\DeclareGraphicsExtensions{.pdf,.jpeg,.png} % formatos aceitos

\usepackage{hyphenat} % hífens; veja a última linha antes do início do documento, corrigir erros de hifenização lá!

\usepackage{etoolbox}
\apptocmd{\sloppy}{\hbadness 10000\relax}{}{} % remove erros de underfull hbox em links

\usepackage{color} % permite a mudança de cor do texto

\usepackage{xcolor,soul,framed}
\colorlet{shadecolor}{yellow} % define cor do destacado como amarelo

\usepackage{silence}
\WarningFilter{caption}{Unsupported document class} % remove o erro no pacote 'caption'
\usepackage{caption} % permite a inserção de legendas em imagens

\usepackage{eqparbox} % facilita a criação de grupos

\usepackage{listings} % permite a escrita de códigos de programação

\usepackage{indentfirst} % identação depois de uma seção

\usepackage{tabularx} % tabelas com tamanho variável de coluna

\usepackage{array}

\makeatletter
\patchcmd{\chapter}{\if@openright\cleardoublepage\else\clearpage\fi}{}{}{}
\makeatother % Manter capítulos na mesma página

\hyphenation{matemá-tica} % corrigir erros de hifenização aqui!

\usepackage{placeins}

% Início do documento --------------------------------------------------
\begin{document}

% Cabeçalho
\begin{center}
    \vspace*{-3cm}
    \textsc{Universidade Estacio} \\
    \textsc{Sistemas de Informação}
    
    \vspace{1cm}
    \rule{411pt}{1.3pt}
    \vspace{0.2cm}
    
    \Large \textbf{\textsc{Produto Desenvolvimento}}
    
    \rule{411pt}{1.3pt}
    \vspace{1cm}
    
    Aluno: João Daniel da Silva\\
    \vspace{0.5cm}
  
    
    \vspace{1cm}
\end{center}


% Resumo
\documentclass{article}
\usepackage[utf8]{inputenc}

\title{Relatório Técnico de Produtos IoT}
\author{Seu Nome}
\date{\today}

\begin{document}

\maketitle

\section{Introdução}

A Internet das Coisas (IoT) revolucionou a forma como interagimos com o mundo digital e físico, possibilitando a conexão e comunicação entre dispositivos inteligentes. Nesta seção introdutória, apresentaremos uma visão geral da IoT e sua importância nos setores residencial/comercial e industrial. Além disso, delinearemos o escopo da pesquisa e os objetivos do relatório.

\section{Resumo}

Nesta seção, forneceremos um resumo das descobertas e conclusões do relatório, destacando os principais produtos de IoT identificados, suas características técnicas, aplicações e potenciais impactos nos respectivos setores.

\section{Histórico das Plataformas de IoT}

\subsection{Residencial/Comercial}

Desde o advento da IoT, várias plataformas foram desenvolvidas para atender às demandas de automação e conectividade nos ambientes residenciais e comerciais. Exemplos incluem sistemas de automação residencial baseados em hubs inteligentes, como Amazon Echo, Google Nest, e Apple HomeKit.

\subsection{Industrial}

No setor industrial, plataformas de IoT têm desempenhado um papel crucial na otimização de processos, manutenção preditiva e monitoramento de ativos. Soluções como Siemens MindSphere, IBM Watson IoT e Microsoft Azure IoT oferecem recursos avançados para conectar e gerenciar dispositivos em ambientes industriais complexos.

\section{Produtos de IoT Residencial/Comercial}

\documentclass{article}
\usepackage[utf8]{inputenc}

\title{Descrições de Produtos de IoT}
\author{Seu Nome}
\date{\today}

\begin{document}

\maketitle

\section{Produtos de IoT Residencial/Comercial}

\subsection{Nacionais}

\subsubsection{Produto 1: Smart Home Hub}

\textbf{Descrição:} O Smart Home Hub é um centro de controle inteligente para automação residencial. Ele conecta e gerencia uma variedade de dispositivos inteligentes, como lâmpadas, termostatos, câmeras de segurança e fechaduras, permitindo que os usuários controlem todos os aspectos de sua casa através de um aplicativo móvel intuitivo.

\subsection{Importados}

\subsubsection{Produto 6: Smart Refrigerator}

\textbf{Descrição:} O Smart Refrigerator é uma geladeira inteligente equipada com recursos avançados de conectividade e tecnologia de reconhecimento de voz. Ele permite que os usuários monitorem e controlem remotamente o conteúdo da geladeira, recebam alertas de alimentos vencidos e até mesmo façam compras online diretamente da tela sensível ao toque integrada.

\section{Produtos de IoT Industrial}

\subsection{Nacionais}

\subsubsection{Produto 1: Sistema de Monitoramento de Ativos}

\textbf{Descrição:} O Sistema de Monitoramento de Ativos é uma solução nacional de IoT projetada para empresas industriais. Ele fornece monitoramento em tempo real de ativos, como máquinas, equipamentos e veículos, permitindo a detecção precoce de falhas, manutenção preditiva e otimização de operações para aumentar a eficiência e reduzir custos.

\subsection{Importados}

\subsubsection{Produto 6: Industrial IoT Gateway}

\textbf{Descrição:} O Industrial IoT Gateway é um dispositivo importado que atua como um gateway de comunicação entre dispositivos industriais e plataformas de IoT. Ele suporta uma ampla gama de protocolos de comunicação, como Modbus, OPC-UA e MQTT, e facilita a integração de sistemas legados, permitindo a coleta e análise de dados para tomada de decisões informadas.



\section{Conclusão}

A pesquisa de produtos de IoT residencial/comercial e industrial revela uma ampla gama de soluções tecnológicas disponíveis para atender às necessidades dos consumidores e das indústrias. A evolução contínua nesse campo promete aumentar a eficiência, segurança e sustentabilidade em diversos setores. À medida que a demanda por conectividade e automação cresce, espera-se que novos produtos e tecnologias surjam para impulsionar ainda mais o avanço da IoT.



% Bibliografia
\bibliography{references}



\section{Bibliografia de IoT}

Nos últimos anos, o desenvolvimento de Internet das Coisas (IoT) tem crescido exponencialmente, impulsionado pela demanda por soluções inteligentes e conectadas em diversos setores. A IoT permite a interconexão de dispositivos físicos, coleta de dados em tempo real e automação de processos, trazendo benefícios significativos para residências, empresas e indústrias.

O desenvolvimento de IoT abrange várias etapas, desde a concepção e prototipagem de dispositivos até a implementação de sistemas de gerenciamento de dados e análise de dados. Algumas das principais áreas de foco incluem:

\begin{itemize}
    \item Desenvolvimento de hardware: Projeto e fabricação de dispositivos inteligentes, sensores e atuadores com capacidade de comunicação.
    \item Desenvolvimento de software: Programação de firmware para dispositivos IoT, desenvolvimento de aplicativos móveis e web para interação com os dispositivos.
    \item Conectividade e comunicação: Implementação de protocolos de comunicação como MQTT, CoAP e HTTP para permitir a troca de dados entre dispositivos e servidores.
    \item Segurança: Incorporação de medidas de segurança, como criptografia e autenticação, para proteger os dados transmitidos e garantir a integridade dos dispositivos.
    \item Integração de sistemas: Desenvolvimento de plataformas e middleware para integrar dispositivos IoT com sistemas legados e outros serviços na nuvem.
\end{itemize}

O desenvolvimento de IoT é um campo multidisciplinar que requer conhecimentos em eletrônica, programação, redes de computadores, segurança da informação e design de sistemas. Com o avanço da tecnologia e a crescente adoção de dispositivos conectados, espera-se que o desenvolvimento de IoT continue a evoluir e gerar impactos significativos em diferentes aspectos da vida moderna.


\end{document}