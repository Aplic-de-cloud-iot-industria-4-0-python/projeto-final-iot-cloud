\documentclass{ufersa}

\usepackage{listings} % Pacote para inclusão de códigos

% Configuração do ambiente para códigos Python
\lstnewenvironment{pythoncode}[1][]{
    \lstset{
        language=Python,
        basicstyle=\scriptsize\ttfamily, % Altere para \scriptsize ou \small
        keywordstyle=\color{blue!70!black}\bfseries,
        stringstyle=\color{orange!70!black},
        commentstyle=\color{green!70!black}\itshape,
        morecomment=[l][\color{magenta}]{\#},
        numbers=left,
        numberstyle=\tiny\color{gray}, % Tamanho das linhas numeradas
        stepnumber=1,
        numbersep=5pt,
        tabsize=4,
        showspaces=false,
        showstringspaces=false,
        breaklines=true,
        frame=tb,
        framexleftmargin=5mm,
        aboveskip=3mm,
        belowskip=3mm,
        captionpos=b,
        #1
    }
}{}

\begin{document}
\maketitle

\tableofcontents

\section{Introdução}
Nesta pesquisa, exploraremos uma variedade de cursos de IoT disponíveis online, tanto gratuitos quanto pagos. Esses cursos abrangem uma variedade de tópicos, desde conceitos básicos de IoT até desenvolvimento de aplicativos e implementações avançadas.

\section{Cursos Online}

\subsection{Especialização em Internet das Coisas (IoT) - Coursera}
\begin{itemize}
    \item \textbf{Descrição}: Uma série de cursos que abordam todos os aspectos da IoT, desde conceitos básicos até aplicativos práticos.
    \item \textbf{Link}: \url{https://www.coursera.org/specializations/iot}
    \item \textbf{Preço}: Gratuito (auditoria) ou pago (certificado)
    \item \textbf{Plataforma}: Coursera
\end{itemize}

\subsection{IoT: Internet of Things - edX}
\begin{itemize}
    \item \textbf{Descrição}: Um curso introdutório sobre IoT oferecido pela edX, cobrindo princípios básicos e aplicativos práticos.
    \item \textbf{Link}: \url{https://www.edx.org/learn/iot-internet-of-things}
    \item \textbf{Preço}: Gratuito (auditoria) ou pago (certificado)
    \item \textbf{Plataforma}: edX
\end{itemize}

\subsection{Cursos de IoT - Udemy}
\begin{itemize}
    \item \textbf{Descrição}: Uma variedade de cursos de IoT oferecidos na plataforma Udemy, cobrindo uma ampla gama de tópicos e níveis de habilidade.
    \item \textbf{Link}: \url{https://www.udemy.com/topic/iot/}
    \item \textbf{Preço}: Variável (alguns gratuitos, outros pagos)
    \item \textbf{Plataforma}: Udemy
\end{itemize}

\section{Cursos no YouTube}

\subsection{Canal do YouTube IoT For All}
\begin{itemize}
    \item \textbf{Descrição}: Um canal do YouTube que oferece uma série de vídeos sobre IoT, abrangendo desde conceitos básicos até tutoriais práticos.
    \item \textbf{Link}: \url{https://www.youtube.com/channel/UC-jrDZvuLK6XqhGYx3H9BGw}
    \item \textbf{Preço}: Gratuito
\end{itemize}

\subsection{Canal do YouTube Learn IoT}
\begin{itemize}
    \item \textbf{Descrição}: Outro canal do YouTube que se concentra em tutoriais e projetos práticos de IoT.
    \item \textbf{Link}: \url{https://www.youtube.com/channel/UCotWxn2BivVYw-3rFGaAAWQ}
    \item \textbf{Preço}: Gratuito
\end{itemize}

\section{Conclusão}
Existem muitas opções disponíveis para quem deseja aprender sobre IoT, desde cursos online em plataformas como Coursera, edX e Udemy até recursos gratuitos no YouTube. A escolha do curso depende dos interesses e necessidades individuais, mas com tantas opções disponíveis, é fácil encontrar recursos de alta qualidade para começar a explorar o mundo da Internet das Coisas.

\end{document}