\documentclass{ufersa}

\usepackage{listings} % Pacote para inclusão de códigos

% Configuração do ambiente para códigos Python
\lstnewenvironment{pythoncode}[1][]{
    \lstset{
        language=Python,
        basicstyle=\scriptsize\ttfamily, % Altere para \scriptsize ou \small
        keywordstyle=\color{blue!70!black}\bfseries,
        stringstyle=\color{orange!70!black},
        commentstyle=\color{green!70!black}\itshape,
        morecomment=[l][\color{magenta}]{\#},
        numbers=left,
        numberstyle=\tiny\color{gray}, % Tamanho das linhas numeradas
        stepnumber=1,
        numbersep=5pt,
        tabsize=4,
        showspaces=false,
        showstringspaces=false,
        breaklines=true,
        frame=tb,
        framexleftmargin=5mm,
        aboveskip=3mm,
        belowskip=3mm,
        captionpos=b,
        #1
    }
}{}

\begin{document}
\maketitle

\section{Introdução}
A Internet das Coisas (IoT) está se tornando cada vez mais importante em diversos setores, e a comunicação eficiente entre dispositivos é essencial para o funcionamento adequado dos sistemas IoT. Nesta pesquisa, exploraremos os protocolos mais comuns utilizados na comunicação entre dispositivos IoT.

\section{Protocolos de Comunicação IoT}
Existem vários protocolos de comunicação disponíveis para dispositivos IoT, cada um com suas próprias características e casos de uso específicos. Alguns dos protocolos mais populares incluem:

\subsection{MQTT (Message Queuing Telemetry Transport)}
O MQTT é um protocolo leve de mensagens, projetado para comunicações entre dispositivos com largura de banda limitada e alta latência. Ele utiliza o modelo de publicação/assinatura, facilitando a comunicação assíncrona entre dispositivos.

\subsection{CoAP (Constrained Application Protocol)}
O CoAP é um protocolo de aplicação web projetado para dispositivos com recursos limitados, como sensores e atuadores IoT. Ele oferece um modelo de comunicação baseado em requisição/resposta, semelhante ao HTTP, mas otimizado para ambientes com restrições de largura de banda e energia.

\subsection{HTTP (Hypertext Transfer Protocol)}
Embora não seja exclusivo para IoT, o HTTP ainda é amplamente utilizado para comunicação entre dispositivos IoT e servidores na nuvem. Ele oferece uma variedade de métodos de requisição, como GET, POST, PUT e DELETE, tornando-o flexível para diferentes tipos de interações.

\section{Comparação de Protocolos}
Para determinar o protocolo mais adequado para uma determinada aplicação IoT, é importante considerar vários fatores, como largura de banda disponível, consumo de energia, latência e requisitos de segurança. A tabela a seguir resume algumas das características-chave dos protocolos discutidos:

\begin{center}
\begin{tabular}{|c|c|c|c|}
\hline
Protocolo & Largura de Banda & Consumo de Energia & Latência \\
\hline
MQTT & Baixa & Baixo & Alta \\
CoAP & Baixa & Baixo & Média \\
HTTP & Alta & Alto & Baixa \\
\hline
\end{tabular}
\end{center}

\section{Conclusão}
A escolha do protocolo de comunicação correto é essencial para o sucesso de um sistema IoT. Ao considerar as características de diferentes protocolos, os desenvolvedores podem selecionar a opção mais adequada para atender aos requisitos específicos de sua aplicação.

\end{document}