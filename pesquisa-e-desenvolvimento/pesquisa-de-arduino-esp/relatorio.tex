\documentclass{ufersa}

\usepackage{listings} % Pacote para inclusão de códigos

% Configuração do ambiente para códigos Python
\lstnewenvironment{pythoncode}[1][]{
    \lstset{
        language=Python,
        basicstyle=\scriptsize\ttfamily, % Altere para \scriptsize ou \small
        keywordstyle=\color{blue!70!black}\bfseries,
        stringstyle=\color{orange!70!black},
        commentstyle=\color{green!70!black}\itshape,
        morecomment=[l][\color{magenta}]{\#},
        numbers=left,
        numberstyle=\tiny\color{gray}, % Tamanho das linhas numeradas
        stepnumber=1,
        numbersep=5pt,
        tabsize=4,
        showspaces=false,
        showstringspaces=false,
        breaklines=true,
        frame=tb,
        framexleftmargin=5mm,
        aboveskip=3mm,
        belowskip=3mm,
        captionpos=b,
        #1
    }
}{}

\begin{document}
\maketitle

\section{ESP32}
O ESP32 é um microcontrolador de baixo custo e baixo consumo de energia, com suporte para conexão Wi-Fi e Bluetooth. Ele oferece recursos avançados de processamento e conectividade, tornando-o ideal para projetos de IoT e aplicações sem fio.

\subsection{Principais Características}
\begin{itemize}
    \item Wi-Fi e Bluetooth integrados
    \item Arquitetura de microcontrolador de 32 bits
    \item Alto poder de processamento
    \item Baixo consumo de energia
\end{itemize}

\subsection{Aplicações}
\begin{itemize}
    \item Internet das Coisas (IoT)
    \item Sistemas de monitoramento remoto
    \item Controle de dispositivos via Wi-Fi/Bluetooth
\end{itemize}

\section{ESP8266}
O ESP8266 é um microcontrolador com capacidade de Wi-Fi integrada, amplamente utilizado em projetos de IoT devido ao seu baixo custo e facilidade de integração com redes sem fio.

\subsection{Principais Características}
\begin{itemize}
    \item Wi-Fi integrado
    \item Baixo custo
    \item Facilidade de programação
    \item Suporte para múltiplos protocolos de rede
\end{itemize}

\subsection{Aplicações}
\begin{itemize}
    \item Monitoramento ambiental remoto
    \item Sistemas de automação residencial
    \item Dispositivos conectados à internet
\end{itemize}

\section{Arduino}
O Arduino é uma plataforma de prototipagem eletrônica de código aberto que utiliza uma linguagem de programação baseada em Wiring. Ele é amplamente utilizado devido à sua simplicidade e facilidade de uso, sendo ideal para iniciantes e projetos de pequena escala.

\subsection{Principais Características}
\begin{itemize}
    \item Plataforma de código aberto
    \item Facilidade de prototipagem
    \item Linguagem de programação baseada em Wiring
    \item Diversos modelos disponíveis (Arduino Uno, Arduino Nano, etc.)
\end{itemize}

\subsection{Aplicações}
\begin{itemize}
    \item Projetos de eletrônica DIY
    \item Automação residencial
    \item Robótica educacional
\end{itemize}

\section{Conclusão}
Tanto o Arduino quanto os microcontroladores ESP32 e ESP8266 oferecem uma ampla gama de possibilidades para projetos eletrônicos e IoT. A escolha entre eles dependerá das necessidades específicas do projeto, incluindo requisitos de conectividade, processamento e custo.

\end{document}